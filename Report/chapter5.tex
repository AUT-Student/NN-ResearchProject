\chapter{پیاده‌سازی و نتایج}

\section{تخمین عمر مفید باقی‌مانده}
در جدول ۵-۱ بخشی از نتایج مربوط به کارتحقیقاتی فالکن و همکاران که مدلی بر پایه ماتع برای تحمین عمر مفید باقی‌مانده وسایل مکانیکی پیشنهاد داده بودند آورده شده است. آن‌ها برای ارزیابی مدل از یک تابع امتیاز که در مقاله‌ای دیگر معرفی شده بود و معیار خطای ریشه میانگین مربعات خطا\LTRfootnote{Rooted Mean Square Error (RMSE} استفاده کرده بودند. هر چه این دو معیار برای یک مدل پایین‌تر باشد آن مدل کاراتر است. 

\begin{table}
\begin{center}
\caption{نتایج کار تحقیقاتی فالکن و همکاران برای تخمین عمر مفید باقی‌مانده\cite{falcon2020neural}}
\begin{tabular}{r|r|r}
\toprule
\textbf{مدل} & \textbf{امتیاز} & \textbf{ریشه میانگین مربعات خطا}
\\
\hline
\hline
\lr{LSTM} & ۳۳۹ & ۱۶/۱۶
\\
\lr{LSTM} + ماتع & ۲۴۲ & ۱۲/۵۰
\\
\bottomrule
\end{tabular}
\end{center}
\end{table}

همانگونه که مشخص است استفاده از ماتع در کنار \lr{LSTM} منجر به کاهش حدود ۲۲ درصدی خطا شده است.