\chapter{جمع‌بندی}
در این پروژه ماتع به عنوان یک نوع خاص از شبکه‌های عصبی بازگشتی معرفی شد؛ شبکه‌ای که امکان استفاده از یک حافظه خارجی را دارد. در این مدل محتوای حافظه خارجی در حین آموزش و استنتاج می‌تواند تغییر کند. بخش دیگر ماتع یعنی کنترل‌گر وظیفه کنترل بر حافظه خارجی را بر عهده دارد.
\\

در ادامه گفتیم که به دلیل پیچیدگی بالای این شبکه در ابتدا این شبکه تنها بر روی تعدادی وظیفه ساختگی قابل استفاده بود و حتی پیاده‌سازی مناسب آن که از سرعت خوبی برخوردار باشد چندان ساده نبود. اما نهایتا این شبکه پیشرفت کرد و افزونه‌های متنوعی برای آن ارائه شد. افزونه‌هایی که از بخشی از توان بالقوه ماتع بهره گرفتند و آن را به یک شبکه کاربردی تبدیل کردند.
\\

هر افزونه بر روی بهبودی متمرکز بود؛ مدل ماتع تکاملی امکان استفاده از یک حافظه نامحدود را فراهم کرد و در عین حال تعیین ساختار بهینه شبکه را به یک الگوریتم تکاملی سپارد. مدل ماتع ابرتکاملی مقیاس‌پذیری آن را بهبود داد. مدل ماتع متوجه با کمک مکانیسم توجه استفاده از حافظه را بهتر کرد. مدل ماتع پویا نیز بهبود قابل ملاحظه‌ای در دسترسی به حافظه ایجاد کرد و قدرت شبکه را تقویت کرد. 
\\

ما در این پروژه تحقیقاتی به برخی از کاربردهای واقعی ماتع یعنی تخمین عمر مفید باقی‌مانده، ردیابی دانش، پاسخ به سوالات دوره‌ای و انواع مسائل دسته‌بندی پرداختیم. با ارائه نتایج دریافتیم که این مدل می‌تواند رقیبی برای مدل‌های یادگیری ماشین موجود باشد. نتایج نشان داد که هر افزونه توانسته است دقت‌های بهتری برای این دسته از مدل‌ها ثبت کند. طبیعتا در آینده امکان توسعه افزونه‌های کارآمدتر وجود خواهد داشت.
\\

محدودیت‌های مختلف در برابر پروژه امکان بررسی برخی از افزونه‌های مهم ماتع نظیر کامپیوتر عصبی متمایز و برخی از کاربردها و مقالات دیگر را فراهم نساخت. طبیعتا در پروژه‌های دیگر می‌توان تحقیقات کامل‌تری را در این زمینه انجام داد؛ تحقیقاتی که نواقص موجود و گام‌های بعدی برای بهبود شبکه را آشکار می‌سازند.